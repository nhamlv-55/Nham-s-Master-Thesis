% \section{Neural networks are the new symbolic heuristics}
Symbolic reasoning predates Computer Science. The word \emph{algorithm} itself camed from the 9th-century mathematician Muhammad ibn Musa al-Khwarizmi, Latinized Algoritmi. Logics itself could be traced back to Aristotle in the 300s BC. It is fair to say that the whole field of Computer Science was born out of symbolic reasoning, with pioneering work such as Hilbert's \emph{Entscheidungsproblem}, Alonzo Church's \emph{Lambda Calculus}, and Alan Turing's \emph{Turing machine}.
A century has passed since the birth of modern Computer Science, and the field is now so diverse in the use of mathematics: Topology and Geometry in Computer Graphics, Probability and Linear Algebra in Deep learning, among others.
Nowadays, symbolic reasoning could still be found in Computer Science in its pure form under the research of Programming Languages, Compilers, Formal Methods, and Automated Reasoning. 

Throughout its history, at the heart of all the symbolic reasoning applications are carefully handcrafted heuristics, tried-and-true by years of human experts' research. While those heuristics work wonderfully, they come with the cost of being too specific to the problem at hand. For example, there is no easy way to transfer all the wisdom learnt in crafting a SAT-solver heuristic or the heuristic itself to create a better heuristic for a CHC-solver. This raises a natural, practical, and to a certain extend, a philosophical question: can a heuristic for a symbolic reasoning system be learnt automatically?

Just ten years ago, only the most ambitous researchers would answer ``yes'' for the above question. While glimpse of learnable heuristics could already be found in groundbreaking work such as TD-Gammon \cite{td-gammon}, the whole research direction was practically halted due to hardware limitation. Then, at the turn of the 2010s, researchers realized that the heaviest workload in Deep learning --- matrix multiplication --- could be done very efficiently using GPU --- an easy to find and cheap hardware component. AlexNet \cite{Krizhevsky:nips12} - one of the first works that demonstrated the scalability of Deep Neural Networks using GPUs - blew every other image recognition methods at the time out of the water, and overnight, old ideas were new again: Convolution Neural Network, Long-Short Term Memory (LSTM) Network, Deep Reinforcement Learning, etc. are all in the realm of possiblity. Nowadays, it is hard to find a field that is not yet ``transformed'' or ``revolutionized'' by Deep learning.

Yet, symbolic reasoning is still one of the less successful applications of Deep learning: while Deep learning has been able to achieve human-level, or even superhuman-level on many tasks, such as Image Recognition, Automated Speech Recognition, Game Playing (Go, Atari), it still fails short of the state-of-the-art heuristics at many symbolic reasoning tasks, such as solving mathematical equations \cite{lample2019deep} or solving SAT problems \cite{neuralsat}. Ironically, symbolic reasoning was also the reason for the first \textbf{AI Winter}: Marvin Minsky, founder of the MIT AI Lab, and Seymour Papert, director of the lab at the time, in 1969 published the seminal book Perceptrons \cite{perceptrons}, that discussed perceptron's inability to learn the simple boolean function XOR because it is not linearly separable, turning research away from the perceptron. The history of symbolic reasoning and Deep learning, as often said, comes full circle.

We hypothesize that it is possible to learn Deep learning-based heuristics those are better than handcrafed one. This thesis offers a glimpse into this possiblity, by presenting two concrete positive results where effective heuristics are indeed learnable from data, in two particular domains: compiler optimization, and automated reasoning. We choose to tackle those two domains because we believe the handcrafted heuristics there are not optimal and there are still a lot of room for improvement, and as shown in the following chapters, we indeed improved them by a large margin.

\section{Symbolic Model Checking using co-occurrence probabilities}
In the automated reasoning domain, we introduce \dpy, a neural-based Symbolic Model Checker (SMC) / Constraint Horn Clauses (CHC) Solver.

Model checking has been widely used in various important areas such as robustness analysis of deep neural networks~\cite{Katz:cav19}, verification of hardware designs~\cite{SMC96}, software verification~\cite{Ball02}, analysis~\cite{ESC-java-02} and testing~\cite{Sheyner:SP02}, parameter synthesis in biology~\cite{Barnat:biology12}, and many others. 
The central challenge of model checking is to find a concise and sound approximation of all possible states a given system may reach, which does not cover any undesired states (i.e. violating given specifications). 
Tremendous processes have been made by innovations in efficient data representations~\cite{BDD}, scalable SAT solvers~\cite{CDCL,chaff,minisat}, and effective heuristics~\cite{CEGAR,BMC,McMillan:cav06}.  
Modern model checkers share a common basis, namely, IC3~\cite{IC3}, of which the key insight is \textit{inductive generalization (inductive generalization)}.
This idea has been generalized to support rich theories~\cite{GPDR} that are crucial for many verification tasks~\cite{Komuravelli:cav13,SeaHorn} beyond hardware verification. 
The generalized IC3 with rich theories, also known as satisfiability checking for Constrained Horn Clauses modulo Theory
(CHC)~\cite{DBLP:conf/birthday/BjornerGMR15}, becomes the core part of a broad range of verification tasks.


% \NL{essentially it is a search problem, and traditionally handcrafted heuristics are used. ML can give better heuristic blabh blah}
% With deep learning models being applied in many real life application, verifying them is very important, and push traditional model checking methods to its boundary.

Existing inductive generalization techniques follow either an enumerative search process \cite{IC3,Bradley:fmcad11} or ad-hoc heuristics \cite{Griggio:CAD16,GSpacer}. 
Heuristics are effective but may demand non-trivial domain-specific (or even problem-specific) expertise. 
In this work, we aim to automatically learn such heuristics from the past successful inductive generalizations. 
We observe that verification problems as well as associated inductive generalizations are not isolated from each other. 
Taking software verification as an example, verifying different properties of the same program involves similar or same inductive generalizations; different versions of programs have similar code base; and
different software may use same coding convention, idioms, library or framework, resulting in similar structures.


% Can inductive generalization heuristics be automatically learned?
% Our approach is inspired by the recent advances in deep learning, which automatically learn non-trivial patterns from raw pixels~\cite{Krizhevsky:nips12} as well as semantic correlations between natural language texts~\cite{Mikolov:nips13}.
A natural solution is to train a deep learning model to directly perform inductive generalization, but this approach also raises many new challenges for deep learning. 
% symbols, 
% structured data,
% semantic reasoning (semantic valid/correct), 
% soundness,
% training data.
First of all, the input and the output of inductive generalization are symbolic expressions, which are \textit{highly structured} with \textit{rich semantics}. 
Slight syntactic variations can lead to dramatic changes in semantics.
Second, more importantly, inductive generalization has to satisfy complicated \textit{semantic constraints}. 
Third, given deep learning models hardly provide any reliable guarantees, how to design a neuro-symbolic system that exhibits \textit{learnability} from past experiences but still preserves \textit{soundness}?
% Fourth, how to learn a neural model through a sequence of complicated logical reasoning which are clearly \textit{not differentiable}?
All these challenges have to be properly addressed in building a neuro-symbolic reasoning framework. 
% In this work, we share our design choices and empirical findings in building 

In \cref{chap:dopey}, we present a neuro-symbolic engine \dpy, which introduces a neural component into symbolic model checking. Specifically, we make the following contributions: 
\begin{itemize}
    % \item we study the performance bottleneck of symbolic model checking and
    % factor out a component that are suitable for learning from the past runs;
    \item we adapt standard deep learning models to effectively represent symbolic expressions by incorporating both syntactic and semantic information;
    \item we design a simple but effective learning objective so that training data can be collected with nearly no changes of existing model checkers; 
    \item our integration algorithm achieves the soundness by design, and in the worst case, the learning component may only hurt the running time performance; 
    \item we implement \dpy on top of \spc, a state-of-the-art CHC-solver, using an efficient client-server architecture;
    \item our empirical evaluations indicate \dpy significantly improves \spc on 
    challenging benchmarks from \chccomp.
\end{itemize}


\section{Compiler Optimization using Reinforcement Learning}
In the compiler optimization domain, we introduce \doccam, a reinforcement-learning--based software debloating tool. 
% The rapid increase of software productivity in the last decades was fueled by the
% extensive use of abstraction and reuse of software components. While this
% enabled building larger and more complex software systems, these gains came at
% the expense of less efficient and less secure software systems, and contribute
% to a troublesome trade-off between productivity and performance/security. When
% an application built using general-purpose components is deployed, the majority
% of the general-purpose functionality is never used. This creates two problems:
% first, as the use of abstraction layers makes it more difficult to optimize the
% software, this has a detrimental impact on performance; second, it increases the
% attack surface for security vulnerabilities.

A \emph{Software Debloater} solves a specific problem: given the known running environment, can we remove as much unused code (\emph{debloat}) in the compiled executables and binaries as possible? A motivating example is the frequently used tool \texttt{ls}. \texttt{ls} has more than 40 arguments, yet for many people, \texttt{ls -lath} covers most of the use-cases. By removing code of the rest of \texttt{ls}'s arguments, a software debloater hopes to create a smaller, and safer \texttt{ls}.

% Ashish: While OCCAM, Trimmer, (and Chisel, via tests that the delta-debugged version must pass) specialize the application to its deployment context, other debloaters, such as Piece-wise, do not.
% An emerging solution to this problem is a set of tools, called \emph{Software
%   Debloaters}, that automatically customize a program to a user-specified
% environment. 

One successful approach for software debloating is based on partial
evaluation (PE)~\cite{pe-book} in which a \emph{partial evaluator} takes a
program and its known input values and produces a \emph{residual}
(or \emph{specialized}) program in which those inputs are replaced with constants.
%
% AG: not sure why soundness is important at this point of intro
% A sound partial evaluator must ensure that running the residual program on the
% remaining inputs always produces the same result as running the original
% program on all of its inputs.
%
While PE has been extensively applied to functional and logic programs, it was
less successful on imperative C/\cpp programs (with a noteable exception of
\cmix~\cite{Andersen94} and \tempo~\cite{Consel98}). With the advent of
\llvm~\cite{llvm}, several new partial evaluators of LLVM bitcode have arisen
during the last few years (e.g., \llpe~\cite{llpe}, \occam~\cite{occam}, and
\trimmer~\cite{trimmer}). These tools leverage \llvm optimizations such as
constant propagation, function inlining, and others to either reduce the size of
the residual program (e.g., \occam and \trimmer) or improve performance (e.g.,
\llpe).

The key step of a PE-based debloater is to estimate whether specializing (or
inlining) a function will result in a smaller (or more secure) residual program.
%Unfortunately, as with most problems in program analysis, this is very difficult.
Specialization naturally increases the size of the code since it adds extra
functions. However, since some inputs in the new functions have been replaced by
constant values, optimizations such as constant propagation, might become
enabled and result in new optimization opportunities that reduce the code size.
Therefore, the decision whether or not a function should be specialized for a
particular call-site is non-trivial. % where the solution must be found
% in a large search space that represents the effects of other specialized call
% sites and all LLVM optimizations applied after those specializations.
% Current PE-based software debloaters use naive heuristics. For example, \occam
% implements two heuristics of ``never specialize'' or ``always specialize'',
% respectively; \trimmer specializes a function only if it is called once in the
% whole program. 

% In this thesis, we present a new approach based on reinforcement
% learning (RL) to automatically infer effective heuristics for
% specializing functions for PE-based software debloating. RL is a natural direction since 

Interestingly, this decision of whether a call-site should be specialized or not resembles a long-horizon Markov Decision Process, which fits nicely into the Reinforcement Learning framework: each decision moves the code (state) from one to another, the decision should depend only on the current state instead of the whole history of transformation, and the quality of the whole process (reward) is only known at the end. 



% However, representing code as RL state is a 

% RL is a good fit for the problem for two reasons. First, code specialization
% resembles a Markov Decision Process: each decision moves the code from
% one state to another, and specialization depends only on the current
% state rather than the history of the transformations.  Second, the
% quality of debloating can only be measured after all specialization
% actions and corresponding compiler optimizations have been
% performed. % Rendering approaches like
% supervised-learning inapplicable.

To realize this insight, we need to solve the challenge of deciding on a good state representation.  While it is tempting to use the source code (or LLVM
intermediate representation (IR)) as a state, this is not
computationally tractable. The IR is typically hundreds of MBs in
size. Instead, an adequate set of features that captures meaningful
information about the code while avoiding state aliasing is
required. In particular, these features must capture the \emph{calling context}
in order to distinguish between call-sites. 

% Our finding is presented in \cref{chap:doccam}, in which we makes the following contributions:

In \cref{chap:doccam}, we present a software debloated \doccam, which learns a heuristic for specialization using Reinforcement Learning. Specifically, we make the following contributions: 
\begin{itemize}
    \item we propose a set of Calling context features that enable RL to find a useful heuristic for PE-based debloating software;
    \item we implementation of our method in a software debloater called \doccam;
    \item we evaluate on the reduction of the
program size and number of possible code-reuse attacks, by comparing
our handcrafted features with features learnt automatically via
embedding LLVM IR into a vector space using \insttovec~\cite{inst2vec}.
\end{itemize}
% The main contributions of this paper are threefold: (1) (2) ; and (3) 
% %
% Our initial evaluation suggests that RL is a viable method for
% developing an effective specialization policy for debloating, and learning with
% handcrafted features is easier than with \insttovec.






% tools, \dpy and \doccam. \dpy is a neural-symbolic symbolic Model Checker (SMC) that learns a heuristic for inductive generalization from co-occurrence probabilities between atoms that do (or do not) appear together in facts (called \emph{lemmas}). \doccam is a reinforcement learning based approach to
%   automatically learn a heuristic for optimzing the number of exploitable Return-Oriented Programming (ROP) gadgets during the compilation time.

% Model checking and compiler are as nearly as old as Computer Science are interesting research problems, and important practical applications. Seminar works on compiler and verification can be traced back to the pioneering works of Turing \cite{turing49} and Hopper in late 40s and the early 50s, at the very dawn of modern computing. Verifying the safety of software systems and optimizing the compiled binaries play indisputable roles in the progress of computing since then, and become even more important nowadays: software powers our internet, controls our infrastructure, monitors our health, and needed to be run even on low-powered devices.

% However, compiler optimization and automated model checking still remain two very challenging problems. At the heart of every model checking and compiler optimization algorithms are \emph{symbolic reasoning systems} with carefully crafted heuristics, customized for each and every project, which are expensive and not transferable. 

% This raises a practical research question: \emph{can an effective heuristic for a symbolic reasoning system be discovered automatically?}


% Why are we doing what we are doing?

% Why now? (Deep learning starts to work)

% It is still an open question

% Why the 2 different things

% Write an email, mention what the thesis gonna be about in the email.

