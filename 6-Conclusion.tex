\chapter{Conclusion}
\label{chap:conclusion}
This thesis presents a glimpse into the feasibility of learning state-of-the-art heuristics for symbolic reasoning tasks, specifically, inductive generalization and software debloating, using neural networks.

For inductive generalization, we proposed a neuro-symbolic system, \dpy, which uses a positive and a negative model to approximate co- and anti-occurrences of literals that appeared in past inductive generalizations to improve the overall symbolic model checking process.
For software debloating, we present \doccam --- an end-to-end tool to learn
specialization heuristics using Reinforcement Learning.

In both cases, our results show that the learned heuristics are better than state-of-the-art handcrafted ones, even when we take into account the communication costs between our neural components in Python and our symbolic engine in C++, in the case of \dpy. To the best of our knowledge, they are both the first neural-symbolic guidance that works well in practice for their tasks.

Looking forward, we want to extend the work presented in this thesis to make it more useful, by trying to learn a better representation for the formulas and LLVM IRs, as well as trying to make the learned heuristics transferable to different problems in the same task. 
