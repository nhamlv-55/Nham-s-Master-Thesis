\section{Deep learning}
\subsection{Perceptron. Multi-layer Perceptron}
\subsection{Recurrent-neural network. Tree LSTM}
\subsection{Reinforcement learning}
% In many tasks, the learning goal is beyond predicting some predefined label for a given input. Instead, the learning goal is to reach some beneficial state after a se- quence of actions following a set of rules from some initial state. Thus, what is really learned is a policy, which predicts an action given a state and the action is ideally optimal towards the ultimate beneficial state. Reinforcement learning is a system- atic methodology of solving this exact problem and has achieved remarkable suc- cesses in many challenging problems, particularly games. Prominent examples are strategical games like Chess [93] and Go [166], and video games like Atari [124], StarCraft [179], and Dota [132].
In many tasks, a dataset of input-output pairs $(\vec{x}, y)$ is too expensive, or even impossible to obtain: imagine having to label all the best moves for half of the number of all possible chess boards! More importantly, in many cases, we do not really need one single correct output, but many possible outputs are equally good, as long as their cumulative effects are the same: as long as we reach the destination in time, it doesn't really matter whether our speed at time $t$ is 40 or 50 km/h. To solve those two problems, we need to have a learning paradigm that optimizes for a global \emph{goal}, while collecting the data all by itself. \emph{Reinforcement learning} is such a paradigm, and has achieved remarkable successes in many difficult tasks, such as Robotics \cite{convai} or playing board games \cite{td-gammon, alphago}.
